% for glossary entry
% @entry{bird,
%     name={bird},
%     description = {feathered animal},
%     see={[see also]{duck,goose}}
% }

% if this bib file does not work, try using \input{file.tex}
% where all the \newabbreviation commands have been inserted
% containing all the definitions

% Gls to capitalize first letter
% GLS for full uppercase
% for abbreviations also
% glsxtrshort for abbreviation
% similar for long, full, and capital configurations, add pl at the end for plurals
% glsentryshort, long, plural (referred to shorts) must be used when in section titles
% glslink to allow the link but use a different text (as for href)


% if you want to use also description for the abbreviations/acronyms, you should use bib2gls and define all the entries in a bib file, which is incompatible with Overleaf
\newacronym{SEU}{SEU}{Single Event Upset}
\newacronym{COTS}{COTS}{Commercial Off-The-Shelf}
\newacronym{FPGA}{FPA}{Field Programmable Gate Array}
\newacronym{ASIC}{ASIC}{Application Specific Integrated Circuit}
\newacronym{CLB}{CLB}{Configurable Logic Block}
\newacronym{LAB}{LAB}{Logic Array Block}
\newacronym{LUT}{LUT}{Look-up Table}
\newacronym{HDL}{HDL}{Hardware Description Language}
\newacronym{CPU}{CPU}{Central Processing Unit}
\newacronym{DSP}{DSP}{Digital Signal Processing}
\newacronym{CMOS}{CMOS}{Complementary Metal-Oxide Semiconductor}
\newacronym{TMR}{TMR}{Triple Module Redundancy}

