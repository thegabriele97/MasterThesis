% !TEX encoding = UTF-8 Unicode
% !TEX TS-program = pdflatex

% toptesti document class
\documentclass[%
    a4paper, % not needed, by default it is a4paper, or also b5paper can be used
    corpo=12pt, % dimension of basic font
    % oneside is generally the way to go
    oneside, % two side optimizes for two-face printing, having chapters open on the right (aka odd numbers), if you don't want blank pages put oneside here
    stile=standard,
    %evenboxes, % not needed, to put supervisors and candidate at the same level
    tipotesi=magistrale,
    numerazioneromana, % roman numbering for appendixes and preambles, up to Table of Contents
    openright, % to force opening on the right for double-sided printing
    cucitura=7mm, % for printing, 7mm should be enough
    %dvipsnames, % for compatibility with xcolor, it does not work
]{toptesi}

%%%%%%%%%%%%%%%%%%%%%%%%%%%%%%%%%%%%%%%%%%%%%%%%%%%%
\usepackage[english]{babel}
\usepackage[utf8]{inputenc}
\usepackage[T1]{fontenc}
\usepackage{lmodern}

\usepackage{hyperref} % must be loaded before glossaries-extra

% bibliography
\usepackage[hyperref=true,backref=true,backend=biber,maxbibnames=9,maxcitenames=2,style=numeric,citestyle=numeric,sorting=none]{biblatex} % hyperref uses links, backref goes back to citations, uses biber as backend, with 9 names at most in bibliography and 2 in citations, citing using numbers, and sorting in citation order
% sorting can be also ydnt for year descending, name, title or ynt for ascending year

\usepackage{adjustbox} % to resize boxes by keeping the same aspect ratio
\usepackage{algorithm} % algorithm environment
\usepackage{algpseudocode} % improved pseudo-code
\usepackage{amsfonts}               %  AMS mathematical fonts
\usepackage{amsmath}
\usepackage{amssymb}                %  AMS mathematical symbols
\usepackage{bm}                     %  black/bold mathematical symbols
\usepackage{booktabs}               %  better tables
\usepackage[labelfont=bf]{caption} % font=footnotesize % to have reduced caption font size
\usepackage{csquotes}
\usepackage{enumitem} %left align the bulleted points
\usepackage{geometry}
%\usepackage{glossaries} % to use acronyms and glossary, it has also glossaries-extra as extension, but commands are different
\usepackage[%
    toc, % puts the link in the ToC
    %record, % to use bib2gls
    abbreviations, % to load abbreviations / acronyms
    nonumberlist, % to avoid printing the numbers of the references in the acronyms page
]{glossaries-extra}
\usepackage{graphicx}               %  post-script images
%\usepackage{iwona} % extra fonts, substitute standard ones
\usepackage{listings} % to insert formatted code
\usepackage{lipsum} % for lorem ipsum text, not needed in the real work
\usepackage{makecell} % to change dimensions of cells, for math cases
\usepackage{mathtools} % for additional commands
\usepackage{mfirstuc} % to have capitalization capabilities
\usepackage[final]{microtype}      % microtypography, final lets latex use it also in bibliography
\usepackage{multirow} % to allow for cells covering more than 1 row in tables
\usepackage{nicefrac}       % compact symbols for 1/2, etc.
%\usepackage[lofdepth,lotdepth]{subfig}
\usepackage{ragged2e} % for justifying text
\usepackage{siunitx} % support for SI units of measurement and number typesetting
\usepackage{subfig}
\usepackage{svg} % for svg support, works only if inkscape is installed, default for Overleaf v2
%\usepackage{subfigure}              %  subfigure compatibility, can be removed if subfig
\usepackage{tabularx} % equal-width columns in tables
\usepackage{textcomp} % extra fonts and symbols
\usepackage{url}            % simple URL typesetting
\usepackage{verbatim} % for extended verbatim support
\usepackage{xcolor} % to define colors and use standard CSS names add dvipsnames as option, but it clashes with xcolor loaded in toptesi, pay attention that if it goes in conflict with tikz/beamer, simply use \documentclass[usenames,dvipsnames]{beamer}, along with other custom options when defining the document class


% \setlength\textwidth{7in}
% \setlength\textheight{10in}
% \setlength\oddsidemargin{(\paperwidth-\textwidth)/2 - 1in}
% \setlength\topmargin{(\paperheight-\textheight
% -\headheight-\headsep-\footskip)/2 - 1in}

% this configuration is close to TopTesi in width
% \newgeometry{a4paper,top=3cm,bottom=3cm,left=3cm,right=3cm,%
% heightrounded}
% margins as for libreoffice writer
\newgeometry{top=2cm,bottom=2cm,left=2cm,right=2cm,%
heightrounded}



% how to change Contents to Table of Contents
\addto\captionsenglish{% Replace "english" with the language you use
  \renewcommand{\contentsname}%
    {Table of Contents}%
}

% to change the name of Abbreviations to Acronyms
% not needed if use use entry types and define those
% \renewcommand{\abbreviationsname}{Acronyms}

% to allow line comments in algorithms
\algnewcommand{\LineComment}[1]{\State \(\triangleright\) #1}

% to declare abs and norm
\DeclarePairedDelimiter\abs{\lvert}{\rvert}%
\DeclarePairedDelimiter\norm{\lVert}{\rVert}%

% Swap the definition of \abs* and \norm*, so that \abs
% and \norm resizes the size of the brackets, and the 
% starred version does not.
\makeatletter
\let\oldabs\abs
\def\abs{\@ifstar{\oldabs}{\oldabs*}}
%
\let\oldnorm\norm
\def\norm{\@ifstar{\oldnorm}{\oldnorm*}}
\makeatother


% change this configuration with your info
% if you need fewer or more supervisors you have to change \relatore command by adding or removing lines in the table in toptesi_config
\newcommand{\thesistitle}{Study and development of fault tolerant operating systems for aerospace applications}
\newcommand{\thesisuniversitylogo}{images/logo/polytechnic-university-of-turin-logo_clipped} % choose your logo
\newcommand{\thesiscandidatename}{Salvatore Gabriele}
\newcommand{\thesiscandidatesurname}{La Greca}
\newcommand{\thesissupervisoronetitle}{prof.}
\newcommand{\thesissupervisoronename}{Luca}
\newcommand{\thesissupervisoronesurname}{Sterpone}
\newcommand{\thesissupervisortwotitle}{prof.}
\newcommand{\thesissupervisortwoname}{NAME}
\newcommand{\thesissupervisortwosurname}{SURNAME}
\newcommand{\thesissupervisorthreetitle}{prof.}
\newcommand{\thesissupervisorthreename}{NAME}
\newcommand{\thesissupervisorthreesurname}{SURNAME}
\newcommand{\thesisdate}{MONTH YEAR}
\newcommand{\thesiscourse}{COURSE}
\newcommand{\thesisuniversity}{UNIVERSITY}
\newcommand{\thesislevel}{LEVEL} % master or bachelor
\newcommand{\thesiscandidatetext}{Candidate}
\newcommand{\thesissupervisortext}{Supervisors}


% fontsize is {size}{spacing}\family
\newcommand {\institutionfont}{\fontsize {22}{30}\scshape}
\newcommand {\divisionfont}{\fontsize {16}{20}\rmfamily}
\newcommand {\pretitlefont}{\fontsize {16}{16}\rmfamily}
\newcommand {\customtitlefont}{\fontsize {21}{28}\scshape}% {iwona}{bx}{n}}
\newcommand {\fixednamesfont}{\fontsize {14}{20}\mdseries}
\newcommand {\namesfont}{\fontsize {14}{20}\bfseries}
\newcommand {\footfont}{\fontsize {15}{18}\rmfamily}


\addbibresource{bibliography.bib}

% to load the glossaries, not needed if using bib2gls
% for glossary entry
% @entry{bird,
%     name={bird},
%     description = {feathered animal},
%     see={[see also]{duck,goose}}
% }

% if this bib file does not work, try using \input{file.tex}
% where all the \newabbreviation commands have been inserted
% containing all the definitions

% Gls to capitalize first letter
% GLS for full uppercase
% for abbreviations also
% glsxtrshort for abbreviation
% similar for long, full, and capital configurations, add pl at the end for plurals
% glsentryshort, long, plural (referred to shorts) must be used when in section titles
% glslink to allow the link but use a different text (as for href)


% if you want to use also description for the abbreviations/acronyms, you should use bib2gls and define all the entries in a bib file, which is incompatible with Overleaf
\newacronym{AI}{AI}{artificial intelligence}
\makeglossaries

\begin{document}

\title{\vspace*{-5mm}\textbf{\thesistitle}\\Summary} % vspace is needed to shift upwards the title
\date{\thesisdate}
\author{\textbf{Candidate}:\\\thesiscandidatename~\thesiscandidatesurname\\
\textbf{Supervisors}:\\\thesissupervisoronetitle~\thesissupervisoronename~\thesissupervisoronesurname\\
\thesissupervisortwotitle~\thesissupervisortwoname~\thesissupervisortwosurname\\
\thesissupervisorthreetitle~\thesissupervisorthreename~\thesissupervisorthreesurname}


\ateneo{\thesisuniversity} % university name
\logosede[5cm]{\thesisuniversitylogo} % logo, square brackets contain the height

\titolo{\thesistitle} % title
%\sottotitolo{Metodo dei satelliti medicei} % subtitle

% place/remove a slash \\ to put the name on the following line or after Master Degree Course
\corsodilaurea{\thesiscourse} % course name


%~251197 % id number is not needed

\candidato{\thesiscandidatename~\textsc{\thesiscandidatesurname}} % candidate

% using tabular we can have more than 1 supervisor under the same column
\relatore{\tabular{@{}l}%
    \xmakefirstuc{\thesissupervisoronetitle}~\thesissupervisoronename~\textsc{\thesissupervisoronesurname}\\[0.4ex]
    \xmakefirstuc{\thesissupervisortwotitle}~\thesissupervisortwoname~\textsc{\thesissupervisortwosurname}\\[0.4ex]
    \xmakefirstuc{\thesissupervisorthreetitle}~\thesissupervisorthreename~\textsc{\thesissupervisorthreesurname}
    \endtabular}
%\terzorelatore{Ciao}

% in this way we have Academic Year without stile=classica, so without lines
%\sedutadilaurea{\textsc{Academic~Year} 2019-2020}% per la laurea magistrale
% for PoliTo there is only month year
\sedutadilaurea{\thesisdate}% per la laurea magistrale
% PhD
%\esamedidottorato{Novembre 1610}
%\ciclodidottorato{XV}

% offset for binding, the smaller the better
%\setbindingcorrection{3mm}


\english% or \italian (default)

\iflanguage{english}{%
	%\retrofrontespizio{This work is subject to the Creative Commons Licence}

	\CorsoDiLaureaIn{\thesislevel's Degree Course in\space}

	\TesiDiLaurea{\thesislevel's Degree Thesis}

	\InName{in}
	\CandidateName{\xmakefirstuc{\thesiscandidatetext}}% or Candidates
	\AdvisorName{\xmakefirstuc{\thesissupervisortext}}% or Supervisor
	%\TutorName{Tutor}
	%\NomeTutoreAziendale{Internship Tutor}

	%\NomePrimoTomo{First volume}
	%\NomeSecondoTomo{Second Volume}
	%\NomeTerzoTomo{Third Volume}
	%\NomeQuartoTomo{Fourth Volume}
}{}


% front page
% frontespizio can be used for the first page print
% while the custom-made frontpage can be used as hard-cover
% use pdfjoin or pdfseparate to extract or put together the pages if needed
%\frontespizio* % without star the logo is on top
\newgeometry{top=4cm,left=3cm,right=3cm,bottom=4cm,heightrounded}
\begin{titlepage}
\centering
%
{\institutionfont \textbf{\MakeUppercase{\thesisuniversity}} \par}
%
\vspace{\stretch{2}} % changing this number and the others changes the proportion
%
{\divisionfont \textbf{\thesislevel's Degree in \thesiscourse} \par}
%
\vspace{\stretch{3}}
%
\includegraphics[width=50mm]{\thesisuniversitylogo}\\
%
\vspace{\stretch{4}}
%
{\divisionfont \textbf{\thesislevel's Degree Thesis} \par}
%
\vspace{\stretch{3}}
%
{\customtitlefont \textbf{\thesistitle} \par}
%
\vspace{\stretch{10}}
%
\makebox[\textwidth]{\null\hfill\def\arraystretch{2}% % to change the spacing change this number
\begin{minipage}[t]{.375\textwidth}\raggedright
    \begin{adjustbox}{width={\textwidth},totalheight={\textheight},keepaspectratio} % with adjustbox it adapts to the lengths of the names, remove it if you want the same font dimension
    \begin{tabular}[t]{@{}l@{}}
        \fixednamesfont \textbf{\thesissupervisortext} \\
        \namesfont \xmakefirstuc{\thesissupervisoronetitle}~\thesissupervisoronename~\MakeUppercase{\thesissupervisoronesurname}\\
        \namesfont \xmakefirstuc{\thesissupervisortwotitle}~\thesissupervisortwoname~\MakeUppercase{\thesissupervisortwosurname}\\
        \namesfont \xmakefirstuc{\thesissupervisorthreetitle}~\thesissupervisorthreename~\MakeUppercase{\thesissupervisorthreesurname}
    \end{tabular}
    \end{adjustbox}
\end{minipage}
%
\hfill
%
\begin{minipage}[t]{.375\textwidth}\raggedleft
\begin{adjustbox}{width={\textwidth},totalheight={\textheight},keepaspectratio} % with adjustbox it adapts to the lengths of the names, remove it if you want the same font dimension
\begin{tabular}[t]{@{}l@{}}
    \fixednamesfont \textbf{\thesiscandidatetext} \\
    \namesfont \thesiscandidatename~\MakeUppercase{\thesiscandidatesurname}
\end{tabular}
\end{adjustbox}
\end{minipage}\hfill\null}\\
%
\vspace{\stretch{5}}
%
{\footfont \textbf{\thesisdate} \par}
%
\end{titlepage}

\restoregeometry
 % custom frontpage
%\retrofrontespizio
% insert text for the back of the front page
% if you insert any remove the following \paginavuota
% either a blank page or a back is needed to have double-sided printing
% pay attention to leave the space for the page

%\paginavuota % clears a page

\frontmatter

% abstract if needed
% \begin{abstract}
%     % abstract, choose between abstract and summary
% \end{abstract}

% to create blank pages for openright in frontmatter
% use one of the following two methods
% 1) use the following three lines
%\phantom{0} % needed otherwise cleardoublepage does not clean the page because it sees it empty
%\cleardoublepage
%\thispagestyle{empty} % to have empty page, without numbers
% 2) or
\paginavuota % to manually create a blank page

\sommario
% !TEX encoding = UTF-8 Unicode
% !TEX TS-program = pdflatex

\documentclass[%
    12pt, % font size
    oneside, % if it should be done for printing on two-sides or not
    a4paper, % format of paper
    notitlepage, % to remove the title page
]{article}

%%%%%%%%%%%%%%%%%%%%%%%%%%%%%%%%%%%%%%%%%%%%%%%%%%%%
\usepackage[english]{babel}
\usepackage[utf8]{inputenc}
\usepackage[T1]{fontenc}
\usepackage{lmodern}

\usepackage{hyperref} % must be loaded before glossaries-extra

% bibliography
\usepackage[hyperref=true,backref=true,backend=biber,maxbibnames=9,maxcitenames=2,style=numeric,citestyle=numeric,sorting=none]{biblatex} % hyperref uses links, backref goes back to citations, uses biber as backend, with 9 names at most in bibliography and 2 in citations, citing using numbers, and sorting in citation order
% sorting can be also ydnt for year descending, name, title or ynt for ascending year

\usepackage{adjustbox} % to resize boxes by keeping the same aspect ratio
\usepackage{algorithm} % algorithm environment
\usepackage{algpseudocode} % improved pseudo-code
\usepackage{amsfonts}               %  AMS mathematical fonts
\usepackage{amsmath}
\usepackage{amssymb}                %  AMS mathematical symbols
\usepackage{bm}                     %  black/bold mathematical symbols
\usepackage{booktabs}               %  better tables
\usepackage[labelfont=bf]{caption} % font=footnotesize % to have reduced caption font size
\usepackage{csquotes}
\usepackage{enumitem} %left align the bulleted points
\usepackage{geometry}
%\usepackage{glossaries} % to use acronyms and glossary, it has also glossaries-extra as extension, but commands are different
\usepackage[%
    toc, % puts the link in the ToC
    %record, % to use bib2gls
    abbreviations, % to load abbreviations / acronyms
    nonumberlist, % to avoid printing the numbers of the references in the acronyms page
]{glossaries-extra}
\usepackage{graphicx}               %  post-script images
%\usepackage{iwona} % extra fonts, substitute standard ones
\usepackage{listings} % to insert formatted code
\usepackage{lipsum} % for lorem ipsum text, not needed in the real work
\usepackage{makecell} % to change dimensions of cells, for math cases
\usepackage{mathtools} % for additional commands
\usepackage{mfirstuc} % to have capitalization capabilities
\usepackage[final]{microtype}      % microtypography, final lets latex use it also in bibliography
\usepackage{multirow} % to allow for cells covering more than 1 row in tables
\usepackage{nicefrac}       % compact symbols for 1/2, etc.
%\usepackage[lofdepth,lotdepth]{subfig}
\usepackage{ragged2e} % for justifying text
\usepackage{siunitx} % support for SI units of measurement and number typesetting
\usepackage{subfig}
\usepackage{svg} % for svg support, works only if inkscape is installed, default for Overleaf v2
%\usepackage{subfigure}              %  subfigure compatibility, can be removed if subfig
\usepackage{tabularx} % equal-width columns in tables
\usepackage{textcomp} % extra fonts and symbols
\usepackage{url}            % simple URL typesetting
\usepackage{verbatim} % for extended verbatim support
\usepackage{xcolor} % to define colors and use standard CSS names add dvipsnames as option, but it clashes with xcolor loaded in toptesi, pay attention that if it goes in conflict with tikz/beamer, simply use \documentclass[usenames,dvipsnames]{beamer}, along with other custom options when defining the document class


%%%%%%%%%%%%%%%%%%%%%%%%%%%%%%%%%%%%%%%%%%%%%%%%%%%%
\usepackage[english]{babel}
\usepackage[utf8]{inputenc}
\usepackage[T1]{fontenc}
\usepackage{lmodern}

\usepackage{hyperref} % must be loaded before glossaries-extra

% bibliography
\usepackage[hyperref=true,backref=true,backend=biber,maxbibnames=9,maxcitenames=2,style=numeric,citestyle=numeric,sorting=none]{biblatex} % hyperref uses links, backref goes back to citations, uses biber as backend, with 9 names at most in bibliography and 2 in citations, citing using numbers, and sorting in citation order
% sorting can be also ydnt for year descending, name, title or ynt for ascending year

\usepackage{adjustbox} % to resize boxes by keeping the same aspect ratio
\usepackage{algorithm} % algorithm environment
\usepackage{algpseudocode} % improved pseudo-code
\usepackage{amsfonts}               %  AMS mathematical fonts
\usepackage{amsmath}
\usepackage{amssymb}                %  AMS mathematical symbols
\usepackage{bm}                     %  black/bold mathematical symbols
\usepackage{booktabs}               %  better tables
\usepackage[labelfont=bf]{caption} % font=footnotesize % to have reduced caption font size
\usepackage{csquotes}
\usepackage{enumitem} %left align the bulleted points
\usepackage{geometry}
%\usepackage{glossaries} % to use acronyms and glossary, it has also glossaries-extra as extension, but commands are different
\usepackage[%
    toc, % puts the link in the ToC
    %record, % to use bib2gls
    abbreviations, % to load abbreviations / acronyms
    nonumberlist, % to avoid printing the numbers of the references in the acronyms page
]{glossaries-extra}
\usepackage{graphicx}               %  post-script images
%\usepackage{iwona} % extra fonts, substitute standard ones
\usepackage{listings} % to insert formatted code
\usepackage{lipsum} % for lorem ipsum text, not needed in the real work
\usepackage{makecell} % to change dimensions of cells, for math cases
\usepackage{mathtools} % for additional commands
\usepackage{mfirstuc} % to have capitalization capabilities
\usepackage[final]{microtype}      % microtypography, final lets latex use it also in bibliography
\usepackage{multirow} % to allow for cells covering more than 1 row in tables
\usepackage{nicefrac}       % compact symbols for 1/2, etc.
%\usepackage[lofdepth,lotdepth]{subfig}
\usepackage{ragged2e} % for justifying text
\usepackage{siunitx} % support for SI units of measurement and number typesetting
\usepackage{subfig}
\usepackage{svg} % for svg support, works only if inkscape is installed, default for Overleaf v2
%\usepackage{subfigure}              %  subfigure compatibility, can be removed if subfig
\usepackage{tabularx} % equal-width columns in tables
\usepackage{textcomp} % extra fonts and symbols
\usepackage{url}            % simple URL typesetting
\usepackage{verbatim} % for extended verbatim support
\usepackage{xcolor} % to define colors and use standard CSS names add dvipsnames as option, but it clashes with xcolor loaded in toptesi, pay attention that if it goes in conflict with tikz/beamer, simply use \documentclass[usenames,dvipsnames]{beamer}, along with other custom options when defining the document class


% \setlength\textwidth{7in}
% \setlength\textheight{10in}
% \setlength\oddsidemargin{(\paperwidth-\textwidth)/2 - 1in}
% \setlength\topmargin{(\paperheight-\textheight
% -\headheight-\headsep-\footskip)/2 - 1in}

% this configuration is close to TopTesi in width
% \newgeometry{a4paper,top=3cm,bottom=3cm,left=3cm,right=3cm,%
% heightrounded}
% margins as for libreoffice writer
\newgeometry{top=2cm,bottom=2cm,left=2cm,right=2cm,%
heightrounded}


% \setlength\textwidth{7in}
% \setlength\textheight{10in}
% \setlength\oddsidemargin{(\paperwidth-\textwidth)/2 - 1in}
% \setlength\topmargin{(\paperheight-\textheight
% -\headheight-\headsep-\footskip)/2 - 1in}

% this configuration is close to TopTesi in width
% \newgeometry{a4paper,top=3cm,bottom=3cm,left=3cm,right=3cm,%
% heightrounded}
% margins as for libreoffice writer
\newgeometry{top=2cm,bottom=2cm,left=2cm,right=2cm,%
heightrounded}



% how to change Contents to Table of Contents
\addto\captionsenglish{% Replace "english" with the language you use
  \renewcommand{\contentsname}%
    {Table of Contents}%
}

% to change the name of Abbreviations to Acronyms
% not needed if use use entry types and define those
% \renewcommand{\abbreviationsname}{Acronyms}

% to allow line comments in algorithms
\algnewcommand{\LineComment}[1]{\State \(\triangleright\) #1}

% to declare abs and norm
\DeclarePairedDelimiter\abs{\lvert}{\rvert}%
\DeclarePairedDelimiter\norm{\lVert}{\rVert}%

% Swap the definition of \abs* and \norm*, so that \abs
% and \norm resizes the size of the brackets, and the 
% starred version does not.
\makeatletter
\let\oldabs\abs
\def\abs{\@ifstar{\oldabs}{\oldabs*}}
%
\let\oldnorm\norm
\def\norm{\@ifstar{\oldnorm}{\oldnorm*}}
\makeatother


% change this configuration with your info
% if you need fewer or more supervisors you have to change \relatore command by adding or removing lines in the table in toptesi_config
\newcommand{\thesistitle}{Study and development of fault tolerant operating systems for aerospace applications}
\newcommand{\thesisuniversitylogo}{images/logo/polytechnic-university-of-turin-logo_clipped} % choose your logo
\newcommand{\thesiscandidatename}{Salvatore Gabriele}
\newcommand{\thesiscandidatesurname}{La Greca}
\newcommand{\thesissupervisoronetitle}{prof.}
\newcommand{\thesissupervisoronename}{Luca}
\newcommand{\thesissupervisoronesurname}{Sterpone}
\newcommand{\thesissupervisortwotitle}{prof.}
\newcommand{\thesissupervisortwoname}{NAME}
\newcommand{\thesissupervisortwosurname}{SURNAME}
\newcommand{\thesissupervisorthreetitle}{prof.}
\newcommand{\thesissupervisorthreename}{NAME}
\newcommand{\thesissupervisorthreesurname}{SURNAME}
\newcommand{\thesisdate}{MONTH YEAR}
\newcommand{\thesiscourse}{COURSE}
\newcommand{\thesisuniversity}{UNIVERSITY}
\newcommand{\thesislevel}{LEVEL} % master or bachelor
\newcommand{\thesiscandidatetext}{Candidate}
\newcommand{\thesissupervisortext}{Supervisors}


\addbibresource{bibliography.bib}

% to load the glossaries, not needed if using bib2gls
% for glossary entry
% @entry{bird,
%     name={bird},
%     description = {feathered animal},
%     see={[see also]{duck,goose}}
% }

% if this bib file does not work, try using \input{file.tex}
% where all the \newabbreviation commands have been inserted
% containing all the definitions

% Gls to capitalize first letter
% GLS for full uppercase
% for abbreviations also
% glsxtrshort for abbreviation
% similar for long, full, and capital configurations, add pl at the end for plurals
% glsentryshort, long, plural (referred to shorts) must be used when in section titles
% glslink to allow the link but use a different text (as for href)


% if you want to use also description for the abbreviations/acronyms, you should use bib2gls and define all the entries in a bib file, which is incompatible with Overleaf
\newacronym{AI}{AI}{artificial intelligence}
\makeglossaries

\begin{document}

\title{\vspace*{-5mm}\textbf{\thesistitle}\\Summary} % vspace is needed to shift upwards the title
\date{\thesisdate}
\author{\textbf{Candidate}:\\\thesiscandidatename~\thesiscandidatesurname\\
\textbf{Supervisors}:\\\thesissupervisoronetitle~\thesissupervisoronename~\thesissupervisoronesurname\\
\thesissupervisortwotitle~\thesissupervisortwoname~\thesissupervisortwosurname\\
\thesissupervisorthreetitle~\thesissupervisorthreename~\thesissupervisorthreesurname}


\title{\vspace*{-5mm}\textbf{\thesistitle}\\Summary} % vspace is needed to shift upwards the title
\date{\thesisdate}
\author{\textbf{Candidate}:\\\thesiscandidatename~\thesiscandidatesurname\\
\textbf{Supervisors}:\\\thesissupervisoronetitle~\thesissupervisoronename~\thesissupervisoronesurname\\
\thesissupervisortwotitle~\thesissupervisortwoname~\thesissupervisortwosurname\\
\thesissupervisorthreetitle~\thesissupervisorthreename~\thesissupervisorthreesurname}


\maketitle

% !TEX encoding = UTF-8 Unicode
% !TEX TS-program = pdflatex

\documentclass[%
    12pt, % font size
    oneside, % if it should be done for printing on two-sides or not
    a4paper, % format of paper
    notitlepage, % to remove the title page
]{article}

%%%%%%%%%%%%%%%%%%%%%%%%%%%%%%%%%%%%%%%%%%%%%%%%%%%%
\usepackage[english]{babel}
\usepackage[utf8]{inputenc}
\usepackage[T1]{fontenc}
\usepackage{lmodern}

\usepackage{hyperref} % must be loaded before glossaries-extra

% bibliography
\usepackage[hyperref=true,backref=true,backend=biber,maxbibnames=9,maxcitenames=2,style=numeric,citestyle=numeric,sorting=none]{biblatex} % hyperref uses links, backref goes back to citations, uses biber as backend, with 9 names at most in bibliography and 2 in citations, citing using numbers, and sorting in citation order
% sorting can be also ydnt for year descending, name, title or ynt for ascending year

\usepackage{adjustbox} % to resize boxes by keeping the same aspect ratio
\usepackage{algorithm} % algorithm environment
\usepackage{algpseudocode} % improved pseudo-code
\usepackage{amsfonts}               %  AMS mathematical fonts
\usepackage{amsmath}
\usepackage{amssymb}                %  AMS mathematical symbols
\usepackage{bm}                     %  black/bold mathematical symbols
\usepackage{booktabs}               %  better tables
\usepackage[labelfont=bf]{caption} % font=footnotesize % to have reduced caption font size
\usepackage{csquotes}
\usepackage{enumitem} %left align the bulleted points
\usepackage{geometry}
%\usepackage{glossaries} % to use acronyms and glossary, it has also glossaries-extra as extension, but commands are different
\usepackage[%
    toc, % puts the link in the ToC
    %record, % to use bib2gls
    abbreviations, % to load abbreviations / acronyms
    nonumberlist, % to avoid printing the numbers of the references in the acronyms page
]{glossaries-extra}
\usepackage{graphicx}               %  post-script images
%\usepackage{iwona} % extra fonts, substitute standard ones
\usepackage{listings} % to insert formatted code
\usepackage{lipsum} % for lorem ipsum text, not needed in the real work
\usepackage{makecell} % to change dimensions of cells, for math cases
\usepackage{mathtools} % for additional commands
\usepackage{mfirstuc} % to have capitalization capabilities
\usepackage[final]{microtype}      % microtypography, final lets latex use it also in bibliography
\usepackage{multirow} % to allow for cells covering more than 1 row in tables
\usepackage{nicefrac}       % compact symbols for 1/2, etc.
%\usepackage[lofdepth,lotdepth]{subfig}
\usepackage{ragged2e} % for justifying text
\usepackage{siunitx} % support for SI units of measurement and number typesetting
\usepackage{subfig}
\usepackage{svg} % for svg support, works only if inkscape is installed, default for Overleaf v2
%\usepackage{subfigure}              %  subfigure compatibility, can be removed if subfig
\usepackage{tabularx} % equal-width columns in tables
\usepackage{textcomp} % extra fonts and symbols
\usepackage{url}            % simple URL typesetting
\usepackage{verbatim} % for extended verbatim support
\usepackage{xcolor} % to define colors and use standard CSS names add dvipsnames as option, but it clashes with xcolor loaded in toptesi, pay attention that if it goes in conflict with tikz/beamer, simply use \documentclass[usenames,dvipsnames]{beamer}, along with other custom options when defining the document class


%%%%%%%%%%%%%%%%%%%%%%%%%%%%%%%%%%%%%%%%%%%%%%%%%%%%
\usepackage[english]{babel}
\usepackage[utf8]{inputenc}
\usepackage[T1]{fontenc}
\usepackage{lmodern}

\usepackage{hyperref} % must be loaded before glossaries-extra

% bibliography
\usepackage[hyperref=true,backref=true,backend=biber,maxbibnames=9,maxcitenames=2,style=numeric,citestyle=numeric,sorting=none]{biblatex} % hyperref uses links, backref goes back to citations, uses biber as backend, with 9 names at most in bibliography and 2 in citations, citing using numbers, and sorting in citation order
% sorting can be also ydnt for year descending, name, title or ynt for ascending year

\usepackage{adjustbox} % to resize boxes by keeping the same aspect ratio
\usepackage{algorithm} % algorithm environment
\usepackage{algpseudocode} % improved pseudo-code
\usepackage{amsfonts}               %  AMS mathematical fonts
\usepackage{amsmath}
\usepackage{amssymb}                %  AMS mathematical symbols
\usepackage{bm}                     %  black/bold mathematical symbols
\usepackage{booktabs}               %  better tables
\usepackage[labelfont=bf]{caption} % font=footnotesize % to have reduced caption font size
\usepackage{csquotes}
\usepackage{enumitem} %left align the bulleted points
\usepackage{geometry}
%\usepackage{glossaries} % to use acronyms and glossary, it has also glossaries-extra as extension, but commands are different
\usepackage[%
    toc, % puts the link in the ToC
    %record, % to use bib2gls
    abbreviations, % to load abbreviations / acronyms
    nonumberlist, % to avoid printing the numbers of the references in the acronyms page
]{glossaries-extra}
\usepackage{graphicx}               %  post-script images
%\usepackage{iwona} % extra fonts, substitute standard ones
\usepackage{listings} % to insert formatted code
\usepackage{lipsum} % for lorem ipsum text, not needed in the real work
\usepackage{makecell} % to change dimensions of cells, for math cases
\usepackage{mathtools} % for additional commands
\usepackage{mfirstuc} % to have capitalization capabilities
\usepackage[final]{microtype}      % microtypography, final lets latex use it also in bibliography
\usepackage{multirow} % to allow for cells covering more than 1 row in tables
\usepackage{nicefrac}       % compact symbols for 1/2, etc.
%\usepackage[lofdepth,lotdepth]{subfig}
\usepackage{ragged2e} % for justifying text
\usepackage{siunitx} % support for SI units of measurement and number typesetting
\usepackage{subfig}
\usepackage{svg} % for svg support, works only if inkscape is installed, default for Overleaf v2
%\usepackage{subfigure}              %  subfigure compatibility, can be removed if subfig
\usepackage{tabularx} % equal-width columns in tables
\usepackage{textcomp} % extra fonts and symbols
\usepackage{url}            % simple URL typesetting
\usepackage{verbatim} % for extended verbatim support
\usepackage{xcolor} % to define colors and use standard CSS names add dvipsnames as option, but it clashes with xcolor loaded in toptesi, pay attention that if it goes in conflict with tikz/beamer, simply use \documentclass[usenames,dvipsnames]{beamer}, along with other custom options when defining the document class


% \setlength\textwidth{7in}
% \setlength\textheight{10in}
% \setlength\oddsidemargin{(\paperwidth-\textwidth)/2 - 1in}
% \setlength\topmargin{(\paperheight-\textheight
% -\headheight-\headsep-\footskip)/2 - 1in}

% this configuration is close to TopTesi in width
% \newgeometry{a4paper,top=3cm,bottom=3cm,left=3cm,right=3cm,%
% heightrounded}
% margins as for libreoffice writer
\newgeometry{top=2cm,bottom=2cm,left=2cm,right=2cm,%
heightrounded}


% \setlength\textwidth{7in}
% \setlength\textheight{10in}
% \setlength\oddsidemargin{(\paperwidth-\textwidth)/2 - 1in}
% \setlength\topmargin{(\paperheight-\textheight
% -\headheight-\headsep-\footskip)/2 - 1in}

% this configuration is close to TopTesi in width
% \newgeometry{a4paper,top=3cm,bottom=3cm,left=3cm,right=3cm,%
% heightrounded}
% margins as for libreoffice writer
\newgeometry{top=2cm,bottom=2cm,left=2cm,right=2cm,%
heightrounded}



% how to change Contents to Table of Contents
\addto\captionsenglish{% Replace "english" with the language you use
  \renewcommand{\contentsname}%
    {Table of Contents}%
}

% to change the name of Abbreviations to Acronyms
% not needed if use use entry types and define those
% \renewcommand{\abbreviationsname}{Acronyms}

% to allow line comments in algorithms
\algnewcommand{\LineComment}[1]{\State \(\triangleright\) #1}

% to declare abs and norm
\DeclarePairedDelimiter\abs{\lvert}{\rvert}%
\DeclarePairedDelimiter\norm{\lVert}{\rVert}%

% Swap the definition of \abs* and \norm*, so that \abs
% and \norm resizes the size of the brackets, and the 
% starred version does not.
\makeatletter
\let\oldabs\abs
\def\abs{\@ifstar{\oldabs}{\oldabs*}}
%
\let\oldnorm\norm
\def\norm{\@ifstar{\oldnorm}{\oldnorm*}}
\makeatother


% change this configuration with your info
% if you need fewer or more supervisors you have to change \relatore command by adding or removing lines in the table in toptesi_config
\newcommand{\thesistitle}{Study and development of fault tolerant operating systems for aerospace applications}
\newcommand{\thesisuniversitylogo}{images/logo/polytechnic-university-of-turin-logo_clipped} % choose your logo
\newcommand{\thesiscandidatename}{Salvatore Gabriele}
\newcommand{\thesiscandidatesurname}{La Greca}
\newcommand{\thesissupervisoronetitle}{prof.}
\newcommand{\thesissupervisoronename}{Luca}
\newcommand{\thesissupervisoronesurname}{Sterpone}
\newcommand{\thesissupervisortwotitle}{prof.}
\newcommand{\thesissupervisortwoname}{NAME}
\newcommand{\thesissupervisortwosurname}{SURNAME}
\newcommand{\thesissupervisorthreetitle}{prof.}
\newcommand{\thesissupervisorthreename}{NAME}
\newcommand{\thesissupervisorthreesurname}{SURNAME}
\newcommand{\thesisdate}{MONTH YEAR}
\newcommand{\thesiscourse}{COURSE}
\newcommand{\thesisuniversity}{UNIVERSITY}
\newcommand{\thesislevel}{LEVEL} % master or bachelor
\newcommand{\thesiscandidatetext}{Candidate}
\newcommand{\thesissupervisortext}{Supervisors}


\addbibresource{bibliography.bib}

% to load the glossaries, not needed if using bib2gls
% for glossary entry
% @entry{bird,
%     name={bird},
%     description = {feathered animal},
%     see={[see also]{duck,goose}}
% }

% if this bib file does not work, try using \input{file.tex}
% where all the \newabbreviation commands have been inserted
% containing all the definitions

% Gls to capitalize first letter
% GLS for full uppercase
% for abbreviations also
% glsxtrshort for abbreviation
% similar for long, full, and capital configurations, add pl at the end for plurals
% glsentryshort, long, plural (referred to shorts) must be used when in section titles
% glslink to allow the link but use a different text (as for href)


% if you want to use also description for the abbreviations/acronyms, you should use bib2gls and define all the entries in a bib file, which is incompatible with Overleaf
\newacronym{AI}{AI}{artificial intelligence}
\makeglossaries

\begin{document}

\title{\vspace*{-5mm}\textbf{\thesistitle}\\Summary} % vspace is needed to shift upwards the title
\date{\thesisdate}
\author{\textbf{Candidate}:\\\thesiscandidatename~\thesiscandidatesurname\\
\textbf{Supervisors}:\\\thesissupervisoronetitle~\thesissupervisoronename~\thesissupervisoronesurname\\
\thesissupervisortwotitle~\thesissupervisortwoname~\thesissupervisortwosurname\\
\thesissupervisorthreetitle~\thesissupervisorthreename~\thesissupervisorthreesurname}


\title{\vspace*{-5mm}\textbf{\thesistitle}\\Summary} % vspace is needed to shift upwards the title
\date{\thesisdate}
\author{\textbf{Candidate}:\\\thesiscandidatename~\thesiscandidatesurname\\
\textbf{Supervisors}:\\\thesissupervisoronetitle~\thesissupervisoronename~\thesissupervisoronesurname\\
\thesissupervisortwotitle~\thesissupervisortwoname~\thesissupervisortwosurname\\
\thesissupervisorthreetitle~\thesissupervisorthreename~\thesissupervisorthreesurname}


\maketitle

% !TEX encoding = UTF-8 Unicode
% !TEX TS-program = pdflatex

\documentclass[%
    12pt, % font size
    oneside, % if it should be done for printing on two-sides or not
    a4paper, % format of paper
    notitlepage, % to remove the title page
]{article}

\input{common/packages}

\input{common/summary/packages.tex}

\input{common/package_config}

\input{common/summary/package_config.tex}

\input{common/new_commands}

\input{common/thesis_info.tex}

\addbibresource{bibliography.bib}

% to load the glossaries, not needed if using bib2gls
\input{glossaries.tex}
\makeglossaries

\begin{document}

\input{common/config}

\input{common/summary/config.tex}

\maketitle

\input{content/summary.tex}

\end{document}


\end{document}


\end{document}


\phantom{0}
\cleardoublepage
\thispagestyle{empty}

\ringraziamenti % acknowledgements
% acknowledgements

This thesis work would not be possible without the support of many people. I would like to thank my supervisors for their support and help in the development of this thesis.\bigskip

Thanks to my partner, Heidi G., for constantly listening to me rant and talk about strange things over and over, and for the sacrifices you have made and shared with me in order to help me pursue this Master's Degree. I also want to express my deep appreciation for Carol who taught me the importance of wit, sound sleep, and playfulness and for her cuddly, fidelity and love.\bigskip

Thanks to my parents, my father Gioacchino L., my mother Maria Carmela G. and my sister Carlotta L., for your endless support, both economically as well as for believing in me with regular encouragement in every step to continue in my path.\bigskip

I also would like to thank all my respected teachers and colleagues in the Control and Computer Engineering department. \bigskip

Without all of you, I could have never reached this point in my life.


\vspace*{5\baselineskip}

\begin{flushright}
    \textit{``Life is a journey, and every journey eventually leads to home.''\\
    Crestfallen Saulden, Dark Souls II}
\end{flushright}


\paginavuota
\tableofcontents

\listoftables % ToC for tables

\listoffigures % ToC for figures

% actually abbreviation is the name used for acronym in glossaries-extra
% title sets the name
% type tells the type of glossary to print
% style overrides the global style
% here we are printing only abbreviations
% printunsrtglossary if using record, otherwise printglossary is ok
\paginavuota
\printunsrtglossary[style=altlist,title=Acronyms,type=\glsxtrabbrvtype]

% also list of symbols here if needed

% to remove all first use occurrences given the presence of the summary
\glsresetall
% to skip all the first use occurrences, using only short forms
% \glsunsetall


\mainmatter

%\part{Prima Parte} % parts division, not needed
% Chapters always open on a right-side page, i.e. odd numbers, so a blank page is inserted if needed
%\cleardoublepage[empty] % to have a fully blank page
% a blank page appears before the first chapter in some configurations, on the last version it doesn't

% list here all the chapters
\nocite{xxx}

\chapter{Introduction}
\label{sec:Introduction}

In the last past few years, the number of missions devoted to the exploration of the universe has increased. Predictions show that the number of missions in the current decade is expected to be almost three times the number of missions in the previous decade, without considering low-cost and low-weight missions like the ones including CubeSats. \bigskip

Due to this increase in the number of missions, the overall number of electronic devices on board has increased, and the job complexity assigned to those devices has increased as well. Nowadays, electronic components are used not only for navigation purposes but also for the analysis and manipulation of data. The most advanced spacecrafts are capable of deciding autonomously the trajectory to follow or applying some complex algorithms to the data collected before sending them back to the ground. \bigskip

Whatever the purpose of a spacecraft, from the smallest one to a complete rover exploring another planet, electronic devices must be tailored to work reliably, even in a complex and harsh environment like space, where there are many disturbances like big temperature variations or radiations, one of the most common causes of failure in the spacecraft and greatest enemy of electronic components. \bigskip

To understand better the problem, we can start with a real-world example, a piece of history. On September 22, 2021, the ESA's INTEGRAL spacecraft autonomously entered into emergency safe mode \cite{ESA:INTEGRAL}. INTEGRAL is a space telescope for observing gamma rays, and it was launched into Earth's orbit in 2002. Something catastrophic was happening for the mission itself: one of the spacecraft's three reaction wheels had switched off without warnings. This caused a ripple effect that brought the satellite to begin to rotate uncontrollably. \bigskip

This episode created a lot of problems for the mission itself, and the team of engineers responsible for the INTEGRAL spacecraft had to deal with it: because the spacecraft was spinning, data from the spacecraft were reaching ground control in a difficult way, and the batteries were quickly discharging because of the missing orientation of the solar panels towards the Sun. ESA was going to lose a 19-year-old space telescope. \bigskip

With only a few hours of energy left to save the mission, the Integral Flight Control Team, together with Flight Dynamics and Ground Station Teams started working on a solution, and with quick thinking and ingenious ideas, they found the cause of the problem and rescued the spacecraft. The root of the problem was radiation. Charged, ionized particles, from the Van Allen belt, caused an SEU in the control system of the spacecraft, deciding erroneously to shut down the reaction wheel. \bigskip

This story is an example of the problems that can happen during space missions due to radiation affecting the onboard electronics. From this example, we can understand how crucial is fault-tolerant analysis during all the stages of development of a new space component, in order to produce a dependable system. The concept of a dependable system is a complex one, and in space missions, there are mainly three factors that can affect the dependability of a system:

\begin{itemize}
    \item \textit{Reliability}: the probability of a system to work as expected, continuously, in a given period (usually it coincides with the period of the mission itself).
    \item \textit{Availability}: the probability of a system to work as expected at a generic moment in time, in the future.
    \item \textit{Safety}: the ability of a system to work in a given environment, without any risk of serious damage.
\end{itemize}

With the increasing need for protection against unwanted effects caused by radiations, since the first interplanetary mission in the 60s with the Mariner 2 mission, there have been an increasing number of studies and techniques developed to deal with the problem. At the hardware level, there are \textit{hardware mitigation techniques}, where radiation-tolerant components are used. They are usual referred as \textit{radiation-hard} or \textit{rad-hard} for simplicity. In most of the cases, \textit{COTS} (Commercial Off-The-Shelf) hardware \cite{1589186} is used, which is hardware meant to be used in a generic environment, and on top of that logical mitigation techniques \cite{1546456} are used to protect the system from the effects of radiation. The latter solution is easier to implement, and it is more efficient than the former one in terms of costs. 

\section{Thesis Motivation}
The main motivation for the development of this thesis is to develop some techniques to deal with the problem of radiation in space. In particular, the main goal is to investigate the outcome that can occur when SEU faults affect the CPU (in particular a Xilinx Microblaze soft-core, which will be explained in more detail later on) of a system (like the navigation system of a spacecraft), and how to deal with them by applying some innovative ideas to enhance the system's robustness and so the global fault tolerance of the system. \bigskip

Consequently, the goal is to study and develop some techniques to mitigate errors induced in soft-cores by Single Event Upset faults, which are very common, especially in FPGAs. This area of interest is particularly crucial because complex software, like real-time operating systems (for instance, FreeRTOS), running on top of this hardware, that may be faulty, can create uncoverable and dangerous situations \cite{4375152}. \bigskip

The hardware model on which the techniques are based is the FPGA. FPGAs are used on a lot more space missions nowadays than in the past, for all the reasons that make FPGAs better than ASICs, mainly due to their flexibility. Because of the complexity of space missions, flexibility is a key factor in the success of a mission, both during the development and the operational phases. \bigskip

For this thesis, the usage of FPGAs has one big advantage, among other things: randomly generated SEU faults can be injected easily without using any sophisticated \cite{9459804} hardware, a PC is enough. This is crucial in the study of radiation effects: it's possible to develop a systematic way to inject faults, and they can be repeated over time in order to be able to study the effect of the same SEU with different solutions. Obviously, FPGAs meant to be used in space need to undergo a lot of tests \cite{8708253}, for example in facilities where ultra-high heavy-ion test beams are used to see how the FPGA reacts to real radiation effects.

\nocite{NAP24993}
\nocite{tps}
\nocite{hgc}
\chapter{Background}
\label{sec:background}

Before going further in the implemented solutions, it's better to introduce a few background concepts. In particular, concepts about how FPGAs works, what kind of radiations exists and how FPGAs are affected by them.  

\section{Hardware Technology}

\subsection{FPGA Architecture}
\textit{FPGAs} (Field Programmable Gate Arrays) are used in a wide range of applications, from signal processing to machine learning applications. In particular, it is an integrated circuit designed to be general purpose: after manufacturing, it has no funcionalities. It is a hardware that can be programmed to perform specific tasks. \bigskip

It differs from a CPU. A CPU is an already designed hardware that is designed to do only one thing in a very optimized way: execute code, from a pre-defined Instruction Set. In this case, the action of \textit{programming} is referred to the process of writing a series of instructions that the CPU will eventually execute. This is done by exploiting Programming Languges. A FPGA, instead, is like LEGO bricks. Each LEGO brick alone does not have any function or purpose, but when assembled (so put together with other bricks), it can be used to perform a specific task. Here, the action of \textit{programming} is referred to the process of writing a \textit{description} on how all the bricks will be assembled to perform the specific task we want. The description is done exploiting Hardware Description Languages (HDL) like VHDL or Verilog. \bigskip

The basic FPGA design conists of I/O pads (to connect with the outside world), a set of routing channels and a set of LEGO bricks. A LEGO brick in the FPGA is a logic block (and depending on the vendor, it can be called CLB or LAB) that can be programmed to perform a very specific task that in the overall design helps in achieving the goal of the User's Application. 

\begin{figure}[H]
\centering
\includegraphics[width=1.0\linewidth]{images/chapter2/FPGA_cell_example.png}
\caption{Simplified schematic of a FPGA cell}
\label{fig:fpga_cell}
\end{figure}

A basic logic block consists of a few Logic Elements. As shown in figure \ref{fig:fpga_cell}, a Logic Elements is made of LUTs, a Full-Adder (FA), a D-Type Flip Flop and a bunch of multiplexers. This particular architecture can work in two modes: \textit{normal} mode and \textit{arithmetic} mode. Thanks to the Flip Flop, FPGAs can implement operations where some kind of memory is required.\bigskip

Modern FPGAs are very complex and expand upon the above capabilities to include other functionalities in silicon. Having these common functions embedded in the circuit reduces the area required and gives those functions increased speed compared to building them from logical primitives (because are implemented in-silicon, built out of transistor instead of LUTs, so they have ASICs-level performance). Examples of these include multipliers, generic DSP blocks, embedded processors, high speed I/O logic (like PCI/PCI-Express controllers, DRAM Controllers and so on and so forth) and embedded memories. \bigskip

Once the User's Application is designed (i.e. the description of the FPGA is written), the design needs to be mapped onto the FPGA's hardware resources. This is done using the Vendor's specific software and it's in charge of deciding which FPGA's LE is assigned to which subpart of the description and how each LE is configured. Then, all the LEs needs to be connected between themselfs and the I/O pads, and this is done by routing algorithms that decides the best way to connect them. Once all the implementation steps are done, a configuration file is generated that will eventually used to program the FPGA and is called \textit{bitstream}.\bigskip

All the programmable bits (like the content of the LUTs, some multiplexers selection signals or the routing details) are stored in the FPGA in memory elements that are outside the FPGA's funcional blocks (i.e. the one that can be used by the user to implement the application). Those memory elements can be though of as a big array of bits, or a \textit{shift register}. It's the \textit{configuration memory}: it stores the configuration bits of the entire FPGA and is loaded with the bitstream when the FPGA itself is programmed. Most FPGAs rely on an SRAM-based approach to be programmed: this allows to be in-system programmable (so the FPGA chip can be programmed without unmounting it from the board and from the system itself) and re-programmable (can be programmed as many times we want), but require external boot devices. Because the SRAM is a volatile memory, when the FPGA is powered off, the configuration memory content is lost. An external memory where the bitstream can be retrieved is required in order to re-program it. The SRAM approach is based on CMOS.\bigskip

Consequently, FPGAs are alternatives to hard-core CPUs. This means that on a FPGA a CPU can be implemented out of logic primitives (called \textit{soft-core}), alongside with the hardware that is used to implement the application like peripherals, memory and other components. Modern FPGAs supports \textit{at runtime programming}, this lead to the idea of \textit{reconfigurable systems}, where for example a CPU can be reconfigured in order to enable/disable some of its functionalities to suit the task at hand. The concept of \textit{reconfigurable systems} is also used in another manner and will be explained further in the next chapters.

\subsection{FPGAs vs. ASICs in Aerospace Applications}

An \textit{ASIC} (application-specific integrated circuit) is an integrated circuit chip customized for a particular use. ASIC chips are typically fabricated using metal-oxide-semiconductor (MOS) technology. Thanks to the miniaturization of the MOS-based transistors and the improvement in the design tools, the maximum complexity (and hence functionality) possible in an ASIC has grown from 5000 logic gates to over 100 million. \bigskip

This allows to implement entire microprocessors, memories (including ROM, RAM, EEPROM and flash) and other large component in a single chip. Usually, for lower production volumes, FPGAs may be more cost-effective than an ASIC design. This is due to the non-recurring engineering (NRE) cost of an ASIC, that can run into millions of dollars. \bigskip

1. what asics are
2. how they differs from fpgas
3. why fpga are better in aerospace applications (flexibility, needs to certificate only 1 piece of chip and the hardware can change during the development of the system, ability to reprogram from remote during operational phase) or the fact that software and hardware development can be done in parallel instead of sequentially so faster development and test and implementation time


\section{Radiations}
\chapter{Thesis Background}
\label{sec:backgroundthesis}

This chapter is about the background of the thesis, in order to understand better further chapters and as a help and reference to reproduce the results of this thesis in the future. 

\section{PYNQ-Z2 Development Board}
The \textit{PYNQ-Z2} is a development board designed for the Xilinx University Program. It is equipped with a Xilinx ZYNQ 7020 SoC (XC7Z020-1CLG400C), 512 MB of DDR3 RAM and 16 MB of QSPI Flash Storage. The board provides a clock reference thanks to a crystal oscillator with a frequency of 50 MHz. The reference clock is used by the PS and can be provided to the PL too. 

\begin{figure}[H]
\centering
\includegraphics[width=0.7\linewidth]{images/chapter3/PINOUT.pdf}
\caption{Schematic of the PYNQ-Z2 Development Board}
\label{fig:pynqz2}
\end{figure}

The SoC is made of two subparts: a Processing System (PS) and a Programmable Logic (PL). The PS is the main part of the SoC, containing two 650 MHz ARM Cortex-A9 processor, 512 KB L2 Cache, 256 KB On-Chip Memory and other modules like FPUs, Flash Controller, DRAM Controller, GPIOs and so on.

\begin{figure}[H]
\centering
\includegraphics[width=1.0\linewidth]{images/chapter3/zynq.pdf}
\caption{Schematic of ZYNQ 7020 SoC}
\label{fig:zynq7020}
\end{figure}

A schematic is shown in Figure \ref{fig:zynq7020}. The second part is the PL, which consists in a FPGA with the following characteristics:

\begin{itemize}
    \item 13,300 logic slices, each with four 6-input LUTs and 8 flipflops
    \item 630 KB block RAM
    \item 220 DSP slices
    \item On-chip Xilinx analog-to-digital converter (XADC)
\end{itemize}

The PL can access the Processing System's memory space through a High Performance and/or General Purpose AXI Ports. This enables the usage, for example, of the DDR3 RAM and of the On-Chip memory (OCM) from the PL. The board can be programmed through a JTAG interface, which allows to upload a firmware to be executed from the PS or to program the PL via a bitstream. Moreover, it provides a virtual UART interface that can be used as input/output both for the PS and the PL.\bigskip

\begin{bytefield}{24}
    \memsection{0xbfff\_ffff}{0x4000\_0000}{3}{PL}\\
    \memsection{0x3fff\_ffff}{0x0000\_0000}{3}{OCM/DDR}
\end{bytefield}

% \subsection{Configuration Ports}

\section{Xilinx's Microblaze soft-core}




\section{Xilinx FPGA Standard Design Flow}
\subsection{Bitstream Generation}
\subsection{Fundamentals of the Xilinx's Bitstream structure}
\subsection{Software Development}
% talk about Vitis and xsct
\section{Fault Injection Tool}
\chapter{Hello}
\label{sec:hello}

% use [][] to prepend/postpone text to the citation
\cite[Hi][Goofy]{IEEEexample:article_typical}

\si{\kilo\gram\per\second}

% generic figure
\begin{figure}[h]
\centering
\includegraphics[width=.9\linewidth]{images/logo/logoPoliTo_with_name_wrong.png}
\caption{Hi}
\label{fig:hi}
\end{figure}

% use [] to set name for ToC
\section[Extremely long name with manual linebreak which otherwise would not fit the page]{Extremely long name with manual linebreak\\which otherwise would not fit the page} % ok with fontsize=12pt

% list
\begin{enumerate}
    \item A
    \item B
    \item C
\end{enumerate}

% minipage to put two images in the same figure
\begin{figure}[h]
    \centering
    \begin{minipage}[t]{.49\linewidth}
    \begin{figure}[H]
	\centering
	\includegraphics[width=\linewidth]{images/logo/logoPoliTo_with_name_low_quality.jpg}
	\caption{HI}
	\label{fig:c}
    \end{figure}
    \end{minipage}
    \hfill
    \begin{minipage}[t]{.49\linewidth}
    \begin{figure}[H]
	\centering
	% svg inclusion, requires inkscape
	\includesvg[width=\linewidth]{images/artificial_neural_network.svg}
	\caption{SVG}
	\label{fig:svg}
    \end{figure}
    \end{minipage}
\end{figure}

\begin{table}[]
    \centering
    \setcellgapes{3pt}
    \makegapedcells
    \begin{tabular}{|c|c|c}
    \hline
    ReLU & $f(x) = \begin{cases}
	0 & \text{for } x \le 0\\
	x & \text{for } x > 0\end{cases}$ \\ \hline
    Softmax & $f_i(\vec{x}) = \dfrac{e^{x_i}}{\sum_{j=1}^J e^{x_j}} i = 1, ..., J$ \\ \hline
    tanh & $f(x)=\tanh(x)=\dfrac{(e^{x} - e^{-x})}{(e^{x} + e^{-x})}$ \\ \hline
    \end{tabular}
    \caption{Examples of activation functions, operating either element-wise or vector-wise, depending on the function}
    \label{tab:activation_functions}
\end{table}

\begin{equation}
    \label{eq:fully_connected}
    output = f_{activation}\left(\displaystyle\sum_{\#neurons} input_i + bias\right)
\end{equation}

\begin{table}
    \centering
    \begin{adjustbox}{width={0.9\textwidth},totalheight={\textheight},keepaspectratio} % needed if the table overflows the margins, requires adjustbox package
    \setcellgapes{3pt}
    \makegapedcells
    \begin{tabular}{|c|c|}
    \hline
    MSE / L2 Loss / Quadratic Loss & $\dfrac{\sum_{i=1}^{N} \left(y_i - \hat{y}_i\right)^2}{N}$ \\ \hline
    \makecell{(Binary) Cross Entropy \\ (average reduction on higher dimensions)} & $\dfrac{\sum_{i=1}^{N} \sum_{j=1}^{C} \hat{y}_i \log\left(y_{i,j}\right)}{N}$ \\ \hline
    \makecell{Categorical Cross Entropy \\ (sum reduction on higher dimensions)} & $- \sum_{i=1}^{N} \hat{y}_i +  \log\left(\sum_{i=1}^{N} \sum_{j=1}^{C} y_{i,j}\right)$ \\ \hline
    \end{tabular}
    \end{adjustbox} % must be closed before label and caption
    \caption{$y$ is the output of the network, $N$ is the batch size multiplied by the number of outputs (e.g. pixels), $C$ is the number of classes and $\hat{y}$ is the correct output.}
    \label{tab:loss_functions}
\end{table}


\begin{algorithm}
\caption{Adam optimizer algorithm. All operations are element-wise, even powers. Good values for the constants are $\alpha=0.001, \beta_1 = 0.9, \beta_2 = 0.999, \epsilon = 10^{-8}$. $\epsilon$ is needed to guarantee numerical stability.}
\label{alg:adam_optimizer}
\begin{algorithmic}[1]
\Procedure{Adam}{$\alpha, \beta_1, \beta_2, f, \theta_0$}
\LineComment{$\alpha$ is the stepsize}
\LineComment{$\beta_1, \beta_2 \in \left[0, 1\right)$ are the exponential decay rates for the moment estimates}
\LineComment{$f\left(\theta\right)$ is the objective function to optimize}
\LineComment{$\theta_0$ is the initial vector of parameters which will be optimized}
\LineComment{Initialization}
\State $m_0 \gets 0$
\Comment{First moment estimate vector set to 0}
\State $v_0 \gets 0$
\Comment{Second moment estimate vector set to 0}
\State $t \gets 0$
\Comment{Timestep set to 0}
\LineComment{Execution}
\While{$\theta_t$ not converged}
\State $t \gets t + 1$
\Comment{Update timestep}
\LineComment{Gradients are computed w.r.t the parameters to optimize}
\LineComment{using the value of the objective function}
\LineComment{at the previous timestep}
\State $g_t \gets \nabla_\theta f\left(\theta_{t - 1}\right)$
\LineComment{Update of first-moment and second-moment estimates using}
\LineComment{previous value and new gradients, biased}
\State $m_t \gets \beta_1 \cdot m_{t - 1} + \left( 1 - \beta_1 \right) \cdot g_t$
\State $v_t \gets \beta_2 \cdot v_{t - 1} + \left(1 - \beta_2 \right) \cdot g_t^2$
\LineComment{Bias-correction of estimates}
\State $\hat{m}_t \gets \dfrac{m_t}{1 - \beta_1^t}$
\State $\hat{v}_t \gets \dfrac{v_t}{1 - \beta_2^t}$
\State $\theta_t \gets \theta_{t - 1} - \alpha \cdot \dfrac{\hat{m}_t}{\sqrt{\hat{v}_t} + \epsilon}$
\Comment{Update parameters}
\EndWhile
\State \textbf{return} $\theta_t$
\Comment{Optimized parameters are returned}
\EndProcedure
%\end{small}
\end{algorithmic}
\end{algorithm}

% bullet points
\begin{itemize}
    \item A
    \item B
    \item C
\end{itemize}



% \paginavuota % it works even without stile=classica

\appendix
% appendix
\chapter{Galileo}
\label{sec:appendix_galileo}

%\lstinputlisting[]{} % for source code files directly
% lstlisting environment for direct inclusion
\begin{lstlisting}[language=Python]
    import os
    os.system("echo 1")
\end{lstlisting}

% for computational complexity
$\mathcal{O}\left(n\log{n}\right)$

% verbatim
\verb+numpy+



% endnotes here if needed

\phantom{0}
\cleardoublepage
\printbibliography[heading=bibintoc] % heading required to show it in ToC

\end{document}
