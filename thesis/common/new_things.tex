
% facilitates the creation of memory maps. Start address at the bottom,
% end address at the top.
% syntax:
% \memsection{end address}{start address}{height in lines}{text in box}
\newcommand{\memsection}[4]{%
% define the height of the memsection
\bytefieldsetup{bitheight=#3\baselineskip}%
\bitbox[]{10}{%
\texttt{#1}% print end address
\\
% do some spacing
\vspace{#3\baselineskip}
\vspace{-2\baselineskip}
\vspace{-#3pt}
\texttt{#2}% print start address
}%
\bitbox{16}{#4}% print box with caption
}

\definecolor{codegreen}{rgb}{0,0.6,0}
\definecolor{codegray}{rgb}{0.5,0.5,0.5}
\definecolor{codepurple}{rgb}{0.58,0,0.82}
\definecolor{backcolour}{rgb}{0.95,0.95,0.92}

\lstdefinestyle{preformatted}{
    backgroundcolor=\color{backcolour},   
    commentstyle=\color{codegreen},
    keywordstyle=\color{magenta},
    numberstyle=\tiny\color{codegray},
    stringstyle=\color{codepurple},
    basicstyle=\ttfamily\footnotesize,
    breakatwhitespace=false,         
    breaklines=true,                 
    captionpos=b,                    
    keepspaces=true,                 
    numbers=left,                    
    numbersep=5pt,                  
    showspaces=false,                
    showstringspaces=false,
    showtabs=false,                  
    tabsize=2
}