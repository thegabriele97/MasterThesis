\nocite{xxx}

\chapter{Introduction}
\label{sec:Introduction}

In the last past few years, the number of missions devoted to the exploration of the universe has increased. Predictions show that the number of missions in the current decade is expected to be almost three times the number of missions in the previous decade, without considering low-cost and low-weight missions like the ones including CubeSats. \bigskip

Due to this increase in the number of missions, the overall number of electronic devices on board has increased, and the job complexity assigned to those devices has increased as well. Nowadays, electronic components are used not only for navigation purposes but also for the analysis and manipulation of data. The most advanced spacecrafts are capable of deciding autonomously the trajectory to follow or applying some complex algorithms to the data collected before sending them back to the ground. \bigskip

Whatever the purpose of a spacecraft, from the smallest one to a complete rover exploring another planet, electronic devices must be tailored to work reliably, even in a complex and harsh environment like space, where there are many disturbances like big temperature variations or radiations, one of the most common causes of failure in the spacecraft and greatest enemy of electronic components. \bigskip

To understand better the problem, we can start with a real-world example, a piece of history. On September 22, 2021, the ESA's INTEGRAL spacecraft autonomously entered into emergency safe mode \cite{ESA:INTEGRAL}. INTEGRAL is a space telescope for observing gamma rays, and it was launched into Earth's orbit in 2002. Something catastrophic was happening for the mission itself: one of the spacecraft's three reaction wheels had switched off without warnings. This caused a ripple effect that brought the satellite to begin to rotate uncontrollably. \bigskip

This episode created a lot of problems for the mission itself, and the team of engineers responsible for the INTEGRAL spacecraft had to deal with it: because the spacecraft was spinning, data from the spacecraft were reaching ground control in a difficult way, and the batteries were quickly discharging because of the missing orientation of the solar panels towards the Sun. ESA was going to lose a 19-year-old space telescope. \bigskip

With only a few hours of energy left to save the mission, the Integral Flight Control Team, together with Flight Dynamics and Ground Station Teams started working on a solution, and with quick thinking and ingenious ideas, they found the cause of the problem and rescued the spacecraft. The root of the problem was radiation. Charged, ionized particles, from the Van Allen belt, caused an SEU in the control system of the spacecraft, deciding erroneously to shut down the reaction wheel. \bigskip

This story is an example of the problems that can happen during space missions due to radiation affecting the onboard electronics. From this example, we can understand how crucial is fault-tolerant analysis during all the stages of development of a new space component, in order to produce a dependable system. The concept of a dependable system is a complex one, and in space missions, there are mainly three factors that can affect the dependability of a system:

\begin{itemize}
    \item \textit{Reliability}: the probability of a system to work as expected, continuously, in a given period (usually it coincides with the period of the mission itself).
    \item \textit{Availability}: the probability of a system to work as expected at a generic moment in time, in the future.
    \item \textit{Safety}: the ability of a system to work in a given environment, without any risk of serious damage.
\end{itemize}

With the increasing need for protection against unwanted effects caused by radiations, since the first interplanetary mission in the 60s with the Mariner 2 mission, there have been an increasing number of studies and techniques developed to deal with the problem. At the hardware level, there are \textit{hardware mitigation techniques}, where radiation-tolerant components are used. They are usual referred as \textit{radiation-hard} or \textit{rad-hard} for simplicity. In most of the cases, \textit{COTS} (Commercial Off-The-Shelf) hardware \cite{1589186} is used, which is hardware meant to be used in a generic environment, and on top of that logical mitigation techniques \cite{1546456} are used to protect the system from the effects of radiation. The latter solution is easier to implement, and it is more efficient than the former one in terms of costs. 

\section{Thesis Motivation}
The main motivation for the development of this thesis is to develop some techniques to deal with the problem of radiation in space. In particular, the main goal is to investigate the outcome that can occur when SEU faults affect the CPU (in particular a Xilinx Microblaze soft-core, which will be explained in more detail later on) of a system (like the navigation system of a spacecraft), and how to deal with them by applying some innovative ideas to enhance the system's robustness and so the global fault tolerance of the system. \bigskip

Consequently, the goal is to study and develop some techniques to mitigate errors induced in soft-cores by Single Event Upset faults, which are very common, especially in FPGAs. This area of interest is particularly crucial because complex software, like real-time operating systems (for instance, FreeRTOS), running on top of this hardware, that may be faulty, can create uncoverable and dangerous situations \cite{4375152}. \bigskip

The hardware model on which the techniques are based is the FPGA. FPGAs are used on a lot more space missions nowadays than in the past, for all the reasons that make FPGAs better than ASICs, mainly due to their flexibility. Because of the complexity of space missions, flexibility is a key factor in the success of a mission, both during the development and the operational phases. \bigskip

For this thesis, the usage of FPGAs has one big advantage, among other things: randomly generated SEU faults can be injected easily without using any sophisticated \cite{9459804} hardware, a PC is enough. This is crucial in the study of radiation effects: it's possible to develop a systematic way to inject faults, and they can be repeated over time in order to be able to study the effect of the same SEU with different solutions. Obviously, FPGAs meant to be used in space need to undergo a lot of tests \cite{8708253}, for example in facilities where ultra-high heavy-ion test beams are used to see how the FPGA reacts to real radiation effects.

\nocite{NAP24993}
\nocite{tps}
\nocite{hgc}